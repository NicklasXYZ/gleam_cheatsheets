% -----------------------------------------------------------------------------
\subsection*{\rmfamily Comments}

Comments must start with forward slashes.

\begin{lstlisting}[belowskip=-\baselineskip]
  // This is a comment in Gleam

  /// This is comment to document a statement
  const answer: Int = 42

  //// This is comment to document a module
\end{lstlisting}

% -----------------------------------------------------------------------------
\vs
\subsection*{\rmfamily Operators}

In Gleam the following operators are available:

\begin{lstlisting}[belowskip=-\baselineskip]
  ==  // Equal
  !=  // Not equal
  >   // Int greater than
  >.  // Float greater than
  >=  // Int greater or equal
  >=. // Float greater or equal
  <   // Int less than
  <.  // Float less than
  >=  // Int less or equal
  >=. // Less or equal
  &&  // Boolean and
  ||  // Boolean or	
  +   // Int add
  +.  // Float add
  -   // Int subtract
  -.  // Float subtract
  *   // Int multiply
  *.  // Float multiply
  /   // Int divide
  /.  // Float divide
  %   // Modulo	
  |>  // Pipe
\end{lstlisting}


% -----------------------------------------------------------------------------
\vs
\subsection*{\rmfamily Variables \& Data Types}

The \lstinline[language=Gleam, basicstyle=\small]!let! keyword is used 
before variable names.

\begin{lstlisting}[belowskip=-\baselineskip]
  let size = 5_000

  // Re-assignment of the 'size' variable
  let size = size + 1_000
  let size = 1_000_000
\end{lstlisting}

\vs
\subsubsection*{\rmfamily Variable type annotations}

Type annotations can be given when binding variables.

\begin{lstlisting}[belowskip=-\baselineskip]
  let some_float: Float = 1.0
  let some_int: Int = 1  

  // A list contains elements of the same data type
  let some_list: List(Int) = [1, 2, 3]

  // Strings are UTF-8 encoded and use double quotes 
  let some_string: String = "Hellø, world!"

  // Tuples allow elements with different data types
  let some_tuple: #(Int, Float, String) = #(1, 1.0, "1")
\end{lstlisting}

\vs
\subsubsection*{\rmfamily Maps}

Maps can have keys and values of any type, but all keys must be of 
the same type.

% [belowskip=-\baselineskip]
\begin{lstlisting}
  import gleam/map

  map.from_list([#("key1", "val1"), #("key2", "val2")])
\end{lstlisting}

\vs
Furthermore, all values must be of the same type otherwise:

% [belowskip=-\baselineskip]
\begin{lstlisting}
  import gleam/map

  // Type error!
  map.from_list([#("key1", "val1"), #("key2", 2)])  
\end{lstlisting}

There is no map literal syntax in Gleam, and you cannot pattern match on a map.
Maps are generally not used much in Gleam.

% -----------------------------------------------------------------------------
\vs
\subsubsection*{\rmfamily Match operator}

The \lstinline[language=Gleam, basicstyle=\small]!let! keyword and the
\,\lstinline[language=Gleam, basicstyle=\large]!=!\, operator can be used for
pattern matching. For assertions, the 
\lstinline[language=Gleam, basicstyle=\small]!assert! keyword is preferred.

% [belowskip=-\baselineskip]
\begin{lstlisting}
  // Bind element 1 in tuple to name x
  let #(x, _) = #(1, 2)

  // Pattern match with 1 in list and bind 2 to name y
  let [1, y, _] = [1, 2, 3]  

  // Runtime error
  assert [] = [1]

  // Compile error, type mismatch
  assert [z] = "Hello"
\end{lstlisting}

% \vs
% \subsubsection*{\rmfamily Strings}

% Strings are UTF-8 encoded and specified with double quotes:

% \begin{lstlisting}[belowskip=-\baselineskip]
%   let some_string = "Hellø, world! This is a string!"
% \end{lstlisting}

% \vs
% \subsubsection*{\rmfamily Tuples}

% Tuples are collections of other data types that can be mixed:

% \begin{lstlisting}[belowskip=-\baselineskip]
%   // Specifying a tuple with different data types
%   let my_tuple = #("username", "password", 10)

%   // Unpacking values of a tuple binding a specific
%   // value of the tuple to a name 'password'
%   let #(_, password, _) = my_tuple
% \end{lstlisting}

% \vs
% \subsubsection*{\rmfamily Lists}

% Lists are collections and contain elements of the same type:

% % [belowskip=-\baselineskip]
% \begin{lstlisting}
%   let list = [2, 3, 4]

%   // Compile error, type mismatch
%   [1.0, ..list]
% \end{lstlisting}

\vs
For list destructuring and pattern matching the \,\lstinline[language=Gleam, basicstyle=\Large]!..!\,  
operator can be used.

\begin{lstlisting}[belowskip=-\baselineskip]
  let list = [2, 3, 4]

  // Prepend element 1 to the list
  let list = [1, .. list]

  // Pattern match with 1 in the list and bind the
  // second element to name 'second_element' 
  let [1, second_element, .. ] = list
\end{lstlisting}


% -----------------------------------------------------------------------------
\vs
\subsection*{\rmfamily Functions}

Functions are declared using the 
\lstinline[language=Gleam, basicstyle=\small]!fn! keyword.

\begin{lstlisting}[belowskip=-\baselineskip]
  // This is a named function
  fn sum(x, y) {
    x + y
  }

  // This is an anonymous function 
  let mul = fn(x, y) { x * y }
  mul(1, 2)
\end{lstlisting}

% -----------------------------------------------------------------------------
\vs
\subsubsection*{\rmfamily Exporting functions}

Functions are private by default but can be made public using the 
\lstinline[language=Gleam, basicstyle=\small]!pub! keyword.

\begin{lstlisting}[belowskip=-\baselineskip]
  // This is a private function
  fn mul(x, y) {
    x * y
  }

  // This is a public function
  pub fn sum(x, y) {
    x + y
  }
\end{lstlisting}

% -----------------------------------------------------------------------------
\vs
\subsubsection*{\rmfamily Function type annotations}

Functions can have their argument and return types annotated.

\begin{lstlisting}[belowskip=-\baselineskip]
  pub fn add(x: Int, y: Int) -> Int {
    x + y
  }
  
  // Compile error, return type mismatch  
  pub fn mul(x: Int, y: Int) -> Bool {
    x * y
  }
\end{lstlisting}

% -----------------------------------------------------------------------------
\vs
\subsubsection*{\rmfamily Labelled function arguments}

Arguments can be given a label as well as an internal name.

\begin{lstlisting}[belowskip=-\baselineskip]
  pub fn replace(in string, each pattern, with sub) {
    some_func(string, pattern, sub)
  }

  // Note: Labelled arguments can be given in any order
  replace(each: ",", with: " ", in: "A,B,C")
\end{lstlisting}

% -----------------------------------------------------------------------------
\vs
\subsubsection*{\rmfamily Referencing functions}

A function can be referenced without any special syntax.

\begin{lstlisting}[belowskip=-\baselineskip]
  // Function to be referenced
  pub fn identity(x) {
    x
  }
  
  // Assignment of the function to a variable
  let func = identity
  func(100)
\end{lstlisting}

% -----------------------------------------------------------------------------
\vs
\subsection*{\rmfamily Constants}

Constants can be created using the 
\lstinline[language=Gleam, basicstyle=\small]!const! keyword.

\begin{lstlisting}[belowskip=-\baselineskip]
  const the_answer = 42

  // We can return the value through a function call
  pub fn main() {
    the_answer
  }

  // Constants can also be exported and referenced in 
  // other modules if declared with the 'pub' keyword 
  pub const other_answer = 99
\end{lstlisting}

% -----------------------------------------------------------------------------
\vs
\subsection*{\rmfamily Expression blocks}

Braces \lstinline[language=Gleam, basicstyle=\small]!{ }! are used to group
expressions.

\begin{lstlisting}[belowskip=-\baselineskip]
  pub fn main() {
    let x = {
      some_func(1)
      2
    }
    // Braces change the order of arithmetic operations
    let y = x * {x + 10}
    y
  }
\end{lstlisting}


% -----------------------------------------------------------------------------
\vs
\subsection*{\rmfamily Flow control}

The \lstinline[language=Gleam, basicstyle=\small]!case! expression provides a way
to match a value type to an expression. Some examples are given in the following.

% [belowskip=-\baselineskip]
\begin{lstlisting}
  case some_number {
    0 -> "Zero"
    1 -> "One"
    2 -> "Two"
    // This matches anything
    n -> "Some other number" 
  }
\end{lstlisting}

\vs
A \lstinline[language=Gleam, basicstyle=\small]!case! expression can be coupled
with destructuring to provide native pattern matching.

\begin{lstlisting}
  case xs {
    [] -> "This list is empty"
    [a] -> "This list has 1 element"
    [a, b] -> "This list has 2 elements"
    _other -> "This list has more than 2 elements"
  }
\end{lstlisting}

\vs
A \lstinline[language=Gleam, basicstyle=\small]!case! expression also supports guards.

% NOTE: Remove some vertical space here
\begin{lstlisting}[belowskip=-0.3\baselineskip] 
  case xs {
    // Use the 'if' keyword to add a guard expression
    // to the case clause 
    [a, b, c] if a >. b && a <=. c -> "ok"
    _other -> "ko"
  }
  
\end{lstlisting}

\columnbreak % NOTE: Column break here

\vs
...along with disjoint union matching.

\begin{lstlisting}[belowskip=-\baselineskip]
  case number {
    2 | 4 | 6 | 8 -> "This is an even number"
    1 | 3 | 5 | 7 -> "This is an odd number"
    _ -> "I'm not sure"
  }
\end{lstlisting}

% -----------------------------------------------------------------------------
\vs
\subsubsection*{\rmfamily Error management}

Handling errors means to match the return value of an operation against an
\lstinline[language=Gleam, basicstyle=\small]!Error! or
\lstinline[language=Gleam, basicstyle=\small]!Ok! container.

%[belowskip=-\baselineskip]
\begin{lstlisting}
  case parse_int("123") {
    Error(e) -> io.println("That wasn't an Int")
    Ok(i) -> io.println("We parsed the Int")
  }
\end{lstlisting}

\vs
The \lstinline[language=Gleam, basicstyle=\small]!try! keyword simplifies this.
It will either bind a value to the providing name if 
\lstinline[language=Gleam, basicstyle=\small]!Ok(Value)! is matched, or interrupt
the flow and return \lstinline[language=Gleam, basicstyle=\small]!Error(Value)!.
An example is given in the following.

\begin{lstlisting}[belowskip=-\baselineskip]
  let a_number = "1"
  let an_error = Error("ouch")
  let another_number = "3"
  
  // Successful
  try int_a_number = parse_int(a_number)

  // Error will be returned
  try attempt_int = parse_int(an_error)

  // Consequently, this never gets executed
  try int_another_number = parse_int(another_number)

  // ... and this never gets executed
  Ok(int_a_number + attempt_int + int_another_number)
\end{lstlisting}

% -----------------------------------------------------------------------------
\vs
\subsection*{\rmfamily Type aliases}

Type aliases allow for referencing of arbitrary complex types. Even though 
their type systems do not serve the same function. The \lstinline[language=Gleam, basicstyle=\small]!type!
keyword can be used to create aliases.

\begin{lstlisting}[belowskip=-\baselineskip]
  pub type Headers =
    List(#(String, String))
\end{lstlisting}

% -----------------------------------------------------------------------------
\vs
\subsection*{\rmfamily Custom types}

% -----------------------------------------------------------------------------
\vs
\subsubsection*{\rmfamily Records}

A Custom type allows you to define a collection data type with a fixed number of 
named fields, and the values in those fields can be of differing types.

\begin{lstlisting}[belowskip=-\baselineskip]
  type Person {
    Person(name: String, age: Int)
  }
  
  let person = Person(name: "Jake", age: 35)
  let name = person.name
\end{lstlisting}

% -----------------------------------------------------------------------------
\vs
\subsubsection*{\rmfamily Unions}

Functions must always take and receive one type. To have a union of
two different types they must be wrapped in a new custom type.

\begin{lstlisting}[belowskip=-\baselineskip]
  type IntOrFloat {
    AnInt(Int)
    AFloat(Float)
  }
  
  fn int_or_float(X) {
    case X {
      True -> AnInt(1)
      False -> AFloat(1.0)
    }
  }
\end{lstlisting}

% -----------------------------------------------------------------------------
\vs
\subsubsection*{\rmfamily Opaque custom types}

Custom types can be defined as being opaque, which causes the constructors for
the custom type not to be exported from the module. Without any constructors to
import other modules can only interact with opaque types using the intended API.

\begin{lstlisting}[belowskip=-\baselineskip]
  pub opaque type Identifier {
    Identifier(Int)
  }
  
  // Return 'Identifier' with an assigned value of 100
  pub fn get_id() {
    Identifier(100)
  }
\end{lstlisting}

% -----------------------------------------------------------------------------
\vs
\subsection*{\rmfamily Modules}

A Gleam file is a module that is named based on its filename and directory path.

\begin{lstlisting}[belowskip=-\baselineskip]
  // In file foo.gleam
  pub fn identity(x) {
    x
  }
  
  // In file main.gleam
  import foo

  pub fn main() {
    foo.identity(1)
  }
\end{lstlisting}

% -----------------------------------------------------------------------------
\vs
\subsubsection*{\rmfamily Imports}

Imports are relative to the root 
\lstinline[language=Gleam, basicstyle=\normalsize]!src! folder.
Modules in the same directory will need to reference the entire path
from \lstinline[language=Gleam, basicstyle=\normalsize]!src! for the target module,
even if the target module is in the same folder.

\begin{lstlisting}[belowskip=-\baselineskip]
  // Inside src/nasa/moon_base.gleam
  // Imports src/nasa/rocket_ship.gleam
  import nasa/rocket_ship
  
  pub fn explore_space() {
    rocket_ship.launch()
  }
\end{lstlisting}

% -----------------------------------------------------------------------------
\vs
\subsubsection*{\rmfamily Named Imports}
Named imports can be accomplished using the 
\lstinline[language=Gleam, basicstyle=\small]!as! keyword.

\begin{lstlisting}[belowskip=-\baselineskip]
  import unix/cat
  import animal/cat as kitty  
\end{lstlisting}

% -----------------------------------------------------------------------------
\vs
\subsubsection*{\rmfamily Unqualified imports}

Unqualified imports are done in the following way:

\begin{lstlisting}[belowskip=-\baselineskip]
  import animal/cat.{Cat, stroke}

  pub fn main() {
    let kitty = Cat(name: "Nubi")
    stroke(kitty)
  }
\end{lstlisting}
